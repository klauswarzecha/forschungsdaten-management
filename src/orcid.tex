% Persistent identifiers for scientists
\begin{frame}
    \frametitle{Persistente Identifier für Wissenschaftler}

    \only<1>{    
        \framesubtitle{Herausforderung}
        Jede Ressource ist mit Personen (in unterschiedlichen Rollen) verbunden:

        \begin{block}{Namen als Identifier?}
            \begin{columns}[T]
                \column{0.4\textwidth}
                \begin{table}
                    \caption{Personenverzeichnis Uni Vechta}
                    \begin{tabular}[t]{lc}
                        \toprule
                        Nachname & Häufigkeit \\
                        \midrule
                        Kaiser & 4\\
                        Meyer  & 4\\
                        Müller & 4\\ 
                        \bottomrule
                    \end{tabular}
                \end{table}

                \column{0.5\textwidth}
                \begin{table}
                    \caption{Beliebteste deutsche Vornamen}
                    \begin{tabular}[t]{lll}
                        \toprule
                        Zeitraum & weiblich & männlich\\
                        \midrule
                        1980--1989 & Julia & Christian\\
                        1990--1999 & Anna & Jan\\
                        2000--2009 & Anna & Lu[ck]as\\
                        \bottomrule
                    \end{tabular}
                \end{table}

            \end{columns}
        \end{block}

        \alert{\textbf{Namen sind keine eindeutigen Identifier!}}
    } % end only 1

    \only<2>{  
        \framesubtitle{Anforderungen}

        \textbf{Eindeutige ID für jede beteiligte Person}

        \begin{itemize}
            \item unabhängig von Schreibweise und Wechsel des Namens
            \item unabhängig von Hochschulwechsel
            \item in jedem Stadium des FDM verwendbar
            \item bereits international etabliert (\alert{keine In-House-Lösung})
            \item \textbf{in XML Schema Definition für Metadaten verwendbar}
        \end{itemize}

    } % end only 2

    \only<3> {
        \framesubtitle{ORCID}
        \begin{block}{ORCID (Open Researcher and Contributor ID)}
            \begin{itemize}
                \item Beginn: Oktober 2012; aktuell ca. 3.5 Millionen IDs
                \item kostenlos
                \item Open Source (MIT-Lizenz)
                    % \item unterstützt Publikationslisten (BibTeX)
            \end{itemize}



            bei Publikationen unterstützt oder \textbf{zwingend vorgeschrieben} von:

            \begin{itemize}
                \item American Chemical Society (ACS)
                    % \item American Geophysical Union (AGU)
                    % \item eLife
                \item European Molecular Biology Organization (EMBO) 
                    % \item Hindawi 
                    % \item Institute of Electrical & Electronics Engineers (IEEE) 
                \item Public Library of Science (PLOS) 
                \item Royal Society of Chemistry (RSC)
                \item Scopus (Elsevier, auch nachträglich) 
            \end{itemize}

            \url{orcid.org/0000-0002-1247-4508} Gunther Schmidt (Uni Vechta)
            \url{orcid.org/0000-0001-6912-3234} Klaus-Dieter Warzecha

        \end{block}
    } % end only 3

\end{frame}
